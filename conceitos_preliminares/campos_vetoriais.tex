\section{Campos vetoriais}

\begin{defi}
	Seja $M^n$ uma variedade suave. Um \emph{campo vetorial} é uma aplicação diferenciável $X: M \rightarrow TM$ tal que $\pi \circ X = \text{Id}_M$.
	
\begin{figure}
	\centering
	\begin{tikzcd}
	M \arrow{rd}[swap]{\text{id}} \arrow{r}{X} & TM \arrow{d}{\pi} \\
	& M
	\end{tikzcd}
	\caption{Diagrama comutativo de um campo vetorial}
\end{figure}
	
	onde $\pi: TM \rightarrow M$ denota a projeção canônica
	\begin{equation*}
		\pi(p,v) = p
	\end{equation*}
\end{defi}

\begin{obse}
	A igualdade $\pi \circ X = \text{Id}_M$ significa que $X(p) \in T_p M$ para todo $p \in M$.
\end{obse}

\begin{nota}
	O conjunto dos campos vetoriais $X: M \rightarrow TM$ é denotado por $\mathfrak{X}(M)$.
\end{nota}

\begin{obse}
	Com as operações naturais
	\begin{align*}
		(X+Y)(p) &= X(p) + Y(p)\\
		(cX)(p) &= c X(p)
	\end{align*}
	
	onde $c \in \mathbb{R}$, o conjunto $\mathfrak{X}(M)$ torna-se um espaço vetorial real.
\end{obse}

\begin{obse}
	Dados $X \in \mathfrak{X}(M)$ e uma carta local $(U, \varphi)$ em $M$, podemos escrever
	\begin{align*}
		X(p) = \sum_{i=1}^n a_i (p) \frac{\partial}{\partial x_i} (p), \forall p \in U
	\end{align*}
	
	onde $a_1, \ldots, a_n: U \rightarrow \mathbb{R}$ são funções e $\{ \frac{\partial}{\partial x_1}(p), \ldots, \frac{\partial}{\partial x_n}(p) \}$ é a base de $T_p M$ associada a $\varphi$, i.e.
	\begin{equation*}
		\frac{\partial}{\partial x_i} (p) = (d \varphi)^{-1}(\varphi(p)).e_i
	\end{equation*}
	
	onde $\{ e_1, \ldots, e_n \}$ é a base canônica de $\mathbb{R}^n$.
\end{obse}

Ver que $X \in \mathfrak{X}(M)$ é diferenciável se e somente se os $a_1, \ldots, a_n$ são diferenciáveis em U.