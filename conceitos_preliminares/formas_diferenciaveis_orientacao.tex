\section{Formas diferenciáveis e orientação}

\begin{defi}
	Uma 1-forma $\omega$ em $M$ é uma aplicação diferenciável que a cada $p \in M$ associa um funcional linear $\omega(p): T_p M \rightarrow \mathbb{R}$.
\end{defi}

\begin{defi}
	Uma 2-forma $\omega$ em $M$ é uma aplicação diferenciável que associa a cada $p \in M$ uma 2-forma linear $\omega(p): T_p M \times T_p M \rightarrow \mathbb{R}$ alternada (antissimétrica)
	\begin{equation*}
		\omega(p) (v,w) = - \omega(p) (w,v) 
	\end{equation*}
\end{defi}

\begin{exemplo}
	Analisando $\mathbb{R}^2$ como variedade suave vemos que $T_p \mathbb{R}^2 \cong \mathbb{R}^2$. Além disso, o determinante
	\begin{align*}
		\text{det}: \mathbb{R}^2 \times \mathbb{R}^2 &\rightarrow \mathbb{R}\\
		(v,w) & \mapsto \det \left( \begin{matrix}
		v_1 & w_1\\
		v_2 & w_2
		\end{matrix} \right)
	\end{align*}
	
	é uma 2-forma.
\end{exemplo}

\begin{nota}
	\begin{align*}
		C^{\infty} &= \{ f: M \rightarrow \mathbb{R}: f \text{ é diferenciável} \}\\
		\mathfrak{X}(M) &= \{ X: M \rightarrow TM: X \text{ é um campo vetorial} \}\\
		\mathfrak{X}(M)^* &= \text{1-formas}
	\end{align*}
\end{nota}

\begin{defi}
	Seja $E$ um espaço vetorial com $v,w \in E$ e $f,g \in E^*$. O produto simétrico $\wedge$ está definido por
	\begin{align*}
		(f \wedge g) (v,w) &= f(v) g(w) - f(w) g(v)\\
		&= \text{Área}(v,w)
	\end{align*}
\end{defi}

\begin{defi}
	Uma superfície $M$ é dita \emph{orientável} se existe uma 2-forma $\omega$ em $M$ tal que $\omega(p) \neq 0$ para todo $p \in M$. Fixado uma tal 2-forma $\omega$, dizemos que o par $(M, \omega)$ é uma \emph{superfície orientada}.	
\end{defi}

A existência de uma 2-forma $\omega$ em $M$, com $\omega(p) \neq 0$ para todo $p \in M$, permite-nos decidir se a base $\{ \frac{\partial}{\partial x}(p), \frac{\partial}{\partial y}(p) \}$ do plano tangente $T_p M$, com $p \in U$, é positiva ou negativa (na orientação de $M$). Ou seja, escreva
\begin{equation*}
	\omega(p) = h(p) dx(p) \wedge dy(p)
\end{equation*}

onde $h \in C^{\infty}(U)$. Como $\omega$ não se anula, logo $h$ tem sinal constante em $U$.

Dizemos que:
\begin{itemize}
	\item $(U, \varphi)$ é \emph{orientada positiva} se $h >0$.
	\item $(U, \varphi)$ é \emph{orientada negativa} se $h<0$.
\end{itemize}

\begin{prop}
	Uma superfície $M$ é orientável se y somente se é possível escolher um atlas $\mathcal{A}$ em $M$ tal que o Jacobiano (o determinante da matriz jacobiana) de qualquer mudança de coordenadas é positiva.
\end{prop}

\begin{proof}
	contenidos...
\end{proof}