\documentclass[12pt]{book}
\usepackage[utf8]{inputenc}
\usepackage[portuguese]{babel}
\usepackage{amsmath}
\usepackage{amssymb}
\usepackage{amsfonts}
\usepackage{amsthm}
%\usepackage{import}
\usepackage{enumerate}
\usepackage{mathtools}

\usepackage{tikz}
\usetikzlibrary{cd}

\theoremstyle{definition}
\newtheorem{defi}{Definição}[chapter]

\theoremstyle{definition}
\newtheorem{exemplo}{Exemplo}[chapter]

%\theoremstyle{definition}
%\newtheorem{obs}{Observação}[chapter]

\theoremstyle{definition}
\newtheorem*{nota}{Notação}

\theoremstyle{definition}
\newtheorem*{obse}{Observação}

\theoremstyle{definition}
\newtheorem*{lembrete}{Lembrete}

\theoremstyle{definition}
\newtheorem{prop}{Proposição}[chapter]

\newtheorem{teo}{Teorema}[chapter]
\newtheorem{lema}{Lema}[chapter]
\newtheorem{corolario}{Corolario}[chapter]

\title{Superfícies Mínimas em $S^3$}
\author{Mario Alexis Lamas Espinoza}
\date{ }

\newcommand{\innerproduct}[2]{
\left\langle #1,#2 \right\rangle
}

\newcommand{\pdiff}[1]{
\frac{\partial}{\partial #1}
}

\newcommand{\partialdiff}[2]{
\left( #1 \right)_{#2}
}

\newcommand{\partialdifffrac}[2]{\frac{\partial #1}{\partial #2}}

\newcommand{\conection}[2]{
\nabla_{#1} \left( #2 \right)
}

\newcommand{\christoffel}[2]{
\Gamma_{#1}^{#2}
}

\newcommand{\vectorfields}[1]{
\mathfrak{X}(#1)
}

\newcommand{\liebrackets}[2]{
\left[ #1, #2 \right]
}

\newcommand{\smoothfunction}[1]{
C^{\infty}(#1)
}

\newcommand{\curvaturetensor}[3]{
\mathcal{R} \left( #1,#2 \right) #3
}

\newcommand{\norm}[1]{
\left\| #1 \right\|
}

\newcommand{\realnumbers}{\mathbb{R}}

\newcommand{\complexnumbers}{\mathbb{C}}
%\setlength{\parindent}{4em}
%\setlength{\parskip}{1em}
\linespread{1.3}

\begin{document}

\maketitle

\tableofcontents

\chapter{Conceitos Preliminares}

\import{./conceitos_preliminares/}{variedades_suaves.tex}

\import{./conceitos_preliminares/}{campos_vetoriais.tex}
	
\import{./conceitos_preliminares/}{formas_diferenciaveis_orientacao.tex}	



\import{./conceitos_preliminares/}{variedades_riemannianas.tex}
\chapter{Superfícies Mínimas}

\section{Superfícies Mínimas}

\begin{defi}
	Uma superfície regular $M$ em $\realnumbers^3$ é chamada \emph{superfície mínimas} se $H(p)=0$ para qualquer $p \in M$.
\end{defi}

\begin{obse}
	Se $H \equiv 0$, então $K_1 + K_2 \equiv 0$. Logo $K_1 = -K_2$
\end{obse}

\begin{exemplo}
	Um plano em $\realnumbers^3$ é trivialmente mínima, pois $K_1=K_2=0$.
\end{exemplo}

A motivação histórica do estudo das superfícies mínimas foi dada por Lagrange o ano 1760 como seguinte problema:

Dado uma curva fechada $\gamma$ em $\realnumbers^3$, sem autointerseções, determinar a superfície de área mínima, e que tem $\gamma$ como fronteira.

Seja $M$ uma superfície regular orientada em $\realnumbers^3$, e considere uma função $f \in \smoothfunction{M}$.

\begin{defi}
	Uma \emph{variação normal} de $M$, relativa à função $f$, é uma família de superfícies $M_t$, com $t \in (-\epsilon,\epsilon)$, dadas por:
	\begin{equation*}
		p_t = p + t f(p) N(p),
	\end{equation*}
	
	onde $N$ é o campo unitário normal a $M$, na orientação de $M$.
	
\end{defi}

Para $\epsilon > 0$ suficientemente pequeno, cada conjunto $M_t$ também e uma superfície regular chamada uma \emph{superfície de variação}.

Note que para $t=0$, $M_0=M$. Se $f \equiv 1$, $M_t$ é uma superfície \emph{paralela} a $M$ a uma distancia $t$.

**gráfico**

Dados uma variação normal $M_t$ de $M$ relativa a uma função suave $f: M \rightarrow \realnumbers$, com $t \in (-\epsilon,\epsilon)$, e $D \subset M$ um domínio limitado, considere
\begin{equation*}
	D_t = \{ p_t \in M_t: p \in D \}
\end{equation*}

para cada $t \in (-\epsilon,\epsilon)$, $D_t$ é um domínio correspondente em $M_t$. Definimos em cada $t$
\begin{equation*}
	A(t) = \text{Area}(D_t)
\end{equation*}

\begin{teo}
	\begin{equation*}
		A'(0) = -2 \int_D Hf dA
	\end{equation*}
\end{teo}

A expressão acima chama-se a \emph{formula da primeira variação da área}.

\begin{proof}
	contenidos...
\end{proof}

\begin{prop}
	Uma superfície $M$ em $\realnumbers^3$ é mínima se e somente se $A'(0) = 0$.
\end{prop}

\begin{proof}
	contenidos...
\end{proof}

\begin{obse}
	Suponha que exista uma solução $M$ para o problema de Lagrange, e considere uma variação normal $M_t$ de $M$, com $t \in (-\epsilon,\epsilon)$, dada por uma função suave $f: M \rightarrow \realnumbers$ tal que $f_{|\partial M} = 0$. Como a área de $M$ é mínima temos, em particular, que
	\begin{equation*}
		A(t) \geq A(0)
	\end{equation*}
\end{obse}
\section{A representação de Weistrass}

Considere o plano complexo $\complexnumbers$ identificado com $\realnumbers^2$
\begin{equation*}
	(x,y) \in \realnumbers^2 \mapsto x + iy \in \complexnumbers.
\end{equation*}

Uma função complexa $f: U \subset \complexnumbers \rightarrow \complexnumbers$ pode ser escrita na forma
\begin{equation*}
	f(u,v) = f_1(u,v) + i f_2(u,v)
\end{equation*}

onde $f_1, f_2: U \rightarrow \realnumbers$ são funções reais, denotadas por
\begin{align*}
	f_1 &= \Re(f)\\
	f_2 &= \Im(f)
\end{align*}

tal que $\Re(f)$ é parte real da função $f$ e $\Im(f)$ é a parte imaginaria da função $f$.

\begin{defi}
	Uma função $f: U \subset \complexnumbers \rightarrow \complexnumbers$, definida no aberto $U$, é dita \emph{holomorfa} se $f_1, f_2$ possuem derivadas parciais continuas e satisfazem as equações de Cauchy-Riemann
	\begin{align*}
		\partialdifffrac{f_1}{u} &= \partialdifffrac{f_2}{v}\\
		\partialdifffrac{f_1}{v} &= - \partialdifffrac{f_2}{u}
	\end{align*}
\end{defi}

\begin{defi}
	Uma carta local $(U, \varphi)$ em $M$ é dita \emph{mínima} se $H(p) = 0, \forall p \in \varphi(U)$.
\end{defi}

\begin{corolario}\label{equiv_isoterma_harmonica}
	Seja $(U, \varphi)$ uma carta local isoterma de uma superfície $M \subset \realnumbers^3$. Então $(U, \varphi)$ é minima se e somente se $\varphi$ é harmônica, i.e., $\varphi_{uu} + \varphi_{vv} = 0$.
\end{corolario}

Dadas uma superfície $M \subset \realnumbers^3$ e uma carta local $(U, \varphi)$ em $M$, com
\begin{equation*}
	\varphi(u,v) = (x_1(u,v), x_2(u,v), x_3(u,v)),
\end{equation*}

considere as funções complexas $f_j: U \subset \complexnumbers \rightarrow \complexnumbers, 1 \leq j \leq 3,$ dadas por
\begin{equation}\label{carta_isoterma_cauchy-riemann}
	f_j = \partialdifffrac{x_j}{u} - i \partialdifffrac{x_j}{v}, 1 \leq j \leq 3
\end{equation}

\begin{lema}
	Seja $(U, \varphi)$ uma carta local isoterma em $M$. Então, $\varphi$ é minima se e somente se cada $f_j$, definida em \ref{carta_isoterma_cauchy-riemann}, é holomorfa.
\end{lema}

\begin{proof}
	Pelo corolário \ref{equiv_isoterma_harmonica}, temos que $\varphi$ é minima se e somente se $\varphi$ é harmônica, i.e., $\varphi_{uu} + \varphi_{vv} = 0$. Isso significa que
	\begin{equation*}
		\npartialdifffrac{x_j}{u}{2} + \npartialdifffrac{x_j}{v}{2} = 0, 1 \leq j \leq 3.
	\end{equation*}
	
	Queremos provar que
	\begin{align*}
		\pdiff{u} \Re(f_j) &= \pdiff{v} \Im(f_j),\\
		\pdiff{v} \Re(f_J) &= - \pdiff{u} \Im(f_j)
	\end{align*}
	
	Assim
	\begin{multline*}
		\pdiff{u} \Re(f_J) = \pdiff{u} \partialdifffrac{x_j}{u} = \npartialdifffrac{x_j}{u}{2} = - \npartialdifffrac{x_j}{v}{2} = \pdiff{v} \left( - \partialdifffrac{x_j}{v} \right) = \pdiff{v} \Im(f_j)
	\end{multline*}
	
	Isso prova a 1a equação de Cauchy-Riemann. Por outro lado, como a superfície é regular, vale
	\begin{equation*}
		\varphi_{uv} = \varphi_{vu},
	\end{equation*}
	
	ou seja
	\begin{equation*}
		\frac{\partial^2 x_j}{\partial u \partial v} = \frac{\partial^2 x_j}{\partial v \partial u}.
	\end{equation*}
	
	Assim
	\begin{align*}
		\pdiff{v} \Re(f_j) &= \pdiff{v} \partialdifffrac{x_j}{u} = \pdiff{u} \partialdifffrac{x_j}{v}\\
		&= \pdiff{u} \left( - \Im(f_j) \right)\\
		&= - \pdiff{u} \Im(f_j),
	\end{align*}
	
	que é a 2a equação de Cauchy-Riemann.
\end{proof}

\begin{lema}
	Sejam $M \subset \realnumbers^3$ uma superfície minima e $(U, \varphi)$ uma carta local isoterma. Então, as funções holomorfas $f_j$, definidas em \ref{carta_isoterma_cauchy-riemann}, satisfazem
	\begin{gather}\label{sum_fj_2}
		f_1^2 + f_2^2 + f_3^2 = 0\\ \label{sum_norm_fj_2}
		|f_1|^2 + |f_2|^2 + |f_3|^2 \neq 0
	\end{gather}
	
	Reciprocamente, sejam $f_1, f_2, f_3$ funções holomorfas, definidas num aberto simplesmente conexo $U \subset \complexnumbers$, satisfazendo \ref{sum_fj_2} e \ref{sum_norm_fj_2}. Então, tais funções dão origem a uma carta local isoterma minima $(U, \varphi)$.
\end{lema}

\begin{proof}
	Seja $(U, \varphi)$ a carta local isoterma em $M$. Entao,
	\begin{align*}
		f_1^2 + f_2^2 + f_3^2 &= \sum_{j=1}^{3} \left[ \left( \partialdifffrac{x_j}{u} \right)^2 - \left( \partialdifffrac{x_j}{v} \right)^2 - 2i \partialdifffrac{x_j}{u} \partialdifffrac{x_j}{v} \right]\\
		&= E - G - 2iF = 0,
	\end{align*}
	
	pois $E=G$ e $F=0$. A equação \ref{sum_norm_fj_2} segue da regularidade de $\varphi$, pois $\varphi_u \neq 0$ e $\varphi_v \neq 0$.
\end{proof}

\bibliographystyle{plain}
\bibliography{libros}

\end{document}
