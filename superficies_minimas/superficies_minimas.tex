\section{Superfícies Mínimas}

\begin{defi}
	Uma superfície regular $M$ em $\realnumbers^3$ é chamada \emph{superfície mínimas} se $H(p)=0$ para qualquer $p \in M$.
\end{defi}

\begin{obse}
	Se $H \equiv 0$, então $K_1 + K_2 \equiv 0$. Logo $K_1 = -K_2$
\end{obse}

\begin{exemplo}
	Um plano em $\realnumbers^3$ é trivialmente mínima, pois $K_1=K_2=0$.
\end{exemplo}

A motivação histórica do estudo das superfícies mínimas foi dada por Lagrange o ano 1760 como seguinte problema:

Dado uma curva fechada $\gamma$ em $\realnumbers^3$, sem autointerseções, determinar a superfície de área mínima, e que tem $\gamma$ como fronteira.

Seja $M$ uma superfície regular orientada em $\realnumbers^3$, e considere uma função $f \in \smoothfunction{M}$.

\begin{defi}
	Uma \emph{variação normal} de $M$, relativa à função $f$, é uma família de superfícies $M_t$, com $t \in (-\epsilon,\epsilon)$, dadas por:
	\begin{equation*}
		p_t = p + t f(p) N(p),
	\end{equation*}
	
	onde $N$ é o campo unitário normal a $M$, na orientação de $M$.
	
\end{defi}

Para $\epsilon > 0$ suficientemente pequeno, cada conjunto $M_t$ também e uma superfície regular chamada uma \emph{superfície de variação}.

Note que para $t=0$, $M_0=M$. Se $f \equiv 1$, $M_t$ é uma superfície \emph{paralela} a $M$ a uma distancia $t$.

**gráfico**

Dados uma variação normal $M_t$ de $M$ relativa a uma função suave $f: M \rightarrow \realnumbers$, com $t \in (-\epsilon,\epsilon)$, e $D \subset M$ um domínio limitado, considere
\begin{equation*}
	D_t = \{ p_t \in M_t: p \in D \}
\end{equation*}

para cada $t \in (-\epsilon,\epsilon)$, $D_t$ é um domínio correspondente em $M_t$. Definimos em cada $t$
\begin{equation*}
	A(t) = \text{Area}(D_t)
\end{equation*}

\begin{teo}
	\begin{equation*}
		A'(0) = -2 \int_D Hf dA
	\end{equation*}
\end{teo}

A expressão acima chama-se a \emph{formula da primeira variação da área}.

\begin{proof}
	contenidos...
\end{proof}

\begin{prop}\label{caracteristica_das_superficies_minimas}
	Uma superfície $M$ em $\realnumbers^3$ é mínima se e somente se $A'(0) = 0$.
\end{prop}

\begin{proof}
	contenidos...
\end{proof}

\begin{obse}
	Suponha que exista uma solução $M$ para o problema de Lagrange, e considere uma variação normal $M_t$ de $M$, com $t \in (-\epsilon,\epsilon)$, dada por uma função suave $f: M \rightarrow \realnumbers$ tal que $f_{|\partial M} = 0$. Como a área de $M$ é mínima temos, em particular, que
	\begin{equation*}
		A(t) \geq A(0)
	\end{equation*}
	
	para qualquer $t \in (-\epsilon,\epsilon)$. Portanto, $A'(0)=0$, para toda variação normal $M_t$ de $M$ com $f_{|\partial M}=0$.
	
	Isso mostra, em virtude da proposição \ref{caracteristica_das_superficies_minimas}, que as superfícies de área minima são superfícies minimas no sentido da nossa definição. A reciproca é falsa!
\end{obse}

\begin{prop}
	Não existe superfície minima compacta em $\realnumbers^3$.
\end{prop}

\begin{proof}
	Se $M$ é minima, então
	\begin{equation*}
		H = \frac{1}{2} (k_1 + k_2) = 0.
	\end{equation*}
	
	Logo $k_1 = -k_2$ e se tem que $K = k_1 k_2 \leq 0$.
	
	Se $M$ for compacta, existe $p \in M$ tal que $K(p) > 0$.
\end{proof}

Dada suma superfície regular $M$ em $\realnumbers^3$, considere uma carta local isoterma $(U,\varphi)$ para $M$, i.e., 
\begin{equation*}
	E = G = \lambda^2 \text{ e } F=0,
\end{equation*}

onde $\lambda: U \rightarrow \realnumbers$ é uma função diferenciável com $\lambda > 0$. Note que, nas coordenadas isotermas $\varphi \sim (x,y)$, a curvatura media se expressa como
\begin{align*}
	H &= \frac{eG - 2fF + gE}{2(EG - F^2)}\\
	&= \frac{e + g}{2 \lambda^2}
\end{align*}

\begin{defi}
	Dado uma função diferenciável $f: U \subset \realnumbers^2 \rightarrow \realnumbers$, o \emph{Laplaciano} de $f$, denotado por $\Delta f$, é definido por
	\begin{equation*}
		\Delta f = \frac{\partial^2 f}{\partial x^2} + \frac{\partial^2 f}{\partial y^2}.
	\end{equation*}
	
	Dizemos que $f$ é \emph{harmônica} se $\Delta f = 0$.
	
	Se $(U, \varphi)$ é uma carta local para $M$ como $\varphi = (\varphi_1, \varphi_2, \varphi_3)$, definimos
	\begin{equation*}
		\Delta \varphi = (\Delta \varphi_1, \Delta \varphi_2, \Delta \varphi_3).
	\end{equation*}
\end{defi}

\begin{prop}
	Se $(U, \varphi)$ é uma carta local isoterma em $M$, então 
	\begin{equation*}
		\Delta \varphi = 2 \lambda^2 H N
	\end{equation*}
\end{prop}

\begin{proof}
	Como $\varphi$ é isoterma, com $\varphi \sim (u, v)$, temos
	\begin{gather*}
		\innerproduct{\varphi_u}{\varphi_v} = \lambda^2 = \innerproduct{\varphi_v}{\varphi_u} \\
		\text{ e } \innerproduct{\varphi_u}{\varphi_v} = 0
	\end{gather*}
	
	Derivando, obtemos:
	\begin{align*}
		\innerproduct{\varphi_{uu}}{\varphi_u} &= \innerproduct{\varphi_{vu}}{\varphi_v}\\
		&= - \innerproduct{\varphi_u}{\varphi_{vv}}
	\end{align*}
	
	Disso decorre que
	\begin{equation}\label{eq1}
		\innerproduct{\varphi_{uu} + \varphi_{vv}}{\varphi_u} = 0
	\end{equation}
	
	Analogamente, obtemos:
	\begin{equation}\label{eq2}
		\innerproduct{\varphi_{uu} + \varphi_{vv}}{\varphi_v} = 0
	\end{equation}
	
	De \ref{eq1} e \ref{eq2} concluímos que $\varphi_{uu} + \varphi_{vv}$ é paralela a $N$. Alem disso, como
	\begin{equation}
		H = \frac{e+g}{2 \lambda^2}
	\end{equation}
	
	obtemos:
	\begin{align*}
		2 \lambda^2 H &= e + g = \innerproduct{\varphi_{uu}}{N} + \innerproduct{\varphi_{vv}}{N}\\
		&= \innerproduct{\varphi_{uu} + \varphi_{vv}}{N}.
	\end{align*}
	
	Isso mostra que
	\begin{equation}
		\Delta \varphi = 2 \lambda^2 H N
	\end{equation}
\end{proof}

\begin{corolario}
	Uma superfície $M$ em $\realnumbers^3$ é minima se e somente se toda carta local isoterma é harmônica.
\end{corolario}

\begin{exemplo}
	O \emph{catenoide} é a superfície em $\realnumbers^3$ gerada pela rotação da catenária 
	\begin{equation*}
		y = a \cosh \left( \frac{z}{a} \right)
	\end{equation*}
	
	em torno ao eixo $z$.
	
	(Gráfico)
	
	Assim, o catenoide pode ser parametrizado por
	\begin{equation*}
		\varphi(u,v) = \left( a \cosh v \cos u, a \cosh v \sin u, av \right)
	\end{equation*}
	
	onde $u \in (0, 2 \pi)$ e $v \in \realnumbers$. Para tal $\varphi$, obtemos
	\begin{gather*}
		E = G = a^2 \cosh^2 v,\\
		F = 0,\\
		\varphi_{uu} + \varphi_{vv} = 0.
	\end{gather*}
	
	Portanto o catenoide e uma superfície minima.
\end{exemplo}

\begin{exemplo}
	Considere uma hélice dada por
	\begin{equation*}
		\alpha(u) = \left( \cos u, \sin u, au \right).
	\end{equation*}
	
	Por cada ponto da hélice, trace uma reta paralela ao plano $XY$ que intercepta o eixo $Z$.
	
	(Gráfico)
	
	A superfície gerada por tais retas é o \emph{helicoide} e pode ser parametrizada por
	\begin{equation*}
		\varphi(u,v) = \left( v \cos u, v \sin u, au \right)
	\end{equation*}
	
	com $u \in (0, 2 \pi)$ e $v \in \realnumbers$. Temos
	\begin{gather*}
		E = G = a^2 \cosh^2 v\\
		F = 0\\
		\varphi_{uu} + \varphi_{vv} = 0.
	\end{gather*}
	
	Portanto o helicoide é superfície minima.
\end{exemplo}

\begin{teo}
	Alem do plano,
	\begin{enumerate}[(1)]
		\item O catenoide é a única superfície rotacional minima.
		\item O helicoide é a única superfície regrada minima.
	\end{enumerate}
\end{teo}

\begin{exemplo}
	Dado uma função diferenciável $f: U \rightarrow \realnumbers$, definida num aberto $U \subset \realnumbers^2$, considere o gráfico $\text{Gr}(f)$ de $f$, parametrizado por
	\begin{equation*}
		\varphi(x,y) = (x,y,f(x,y)), (x,y) \in U.
	\end{equation*}
	
	Temos
	\begin{align*}
		\varphi_x &= (1,0,f_x)\\
		\varphi_y &= (0,1,f_y).
	\end{align*}
	
	Assim
	\begin{align*}
		E &= \innerproduct{\varphi_x}{\varphi_x} = 1 + f_x^2\\
		F &= \innerproduct{\varphi_x}{\varphi_y} = f_x f_y\\
		G &= \innerproduct{\varphi_y}{\varphi_y} = 1 + f_y^2.
	\end{align*}
	
	Un campo $n$, normal a $\text{Gr}(f)$, é dado por
	\begin{align*}
		n = \varphi_x \times \varphi_y &= \det \left[ \begin{matrix}
		i & j & k\\
		1 & 0 & f_x\\
		0 & 1 & f_y
		\end{matrix} \right]\\
		&= (-f_x, -f_y, 1).
	\end{align*}
	
	Normalizando, temos
	\begin{equation*}
		N = \frac{n}{\norm{n}} = \frac{1}{\sqrt{1 + f_x^2 + f_y^2}}(-f_x, -f_y, 1).
	\end{equation*}
	
	Como
	\begin{align*}
		\varphi_{xx} &= (0, 0, f_{xx})\\
		\varphi_{xy} &= (0, 0, f_{xy})\\
		\varphi_{yy} &= (0, 0, f_{yy})
	\end{align*}
	
	obtemos
	\begin{align*}
		e &= \innerproduct{\varphi_{xx}}{N} = \frac{f_{xx}}{\sqrt{1 + f_x^2 + f_y^2}}\\
		f &= \innerproduct{\varphi_{xy}}{N} = \frac{f_{xy}}{\sqrt{1 + f_x^2 + f_y^2}}\\
		g &= \innerproduct{\varphi_{yy}}{N} = \frac{f_{yy}}{\sqrt{1 + f_x^2 + f_y^2}}.
	\end{align*}
	
	Assim, como
	\begin{equation*}
		H = \frac{eG - 2fF + gE}{2(EG - F^2)}
	\end{equation*}
	
	segue que se $H \equiv 0$, temos
	\begin{equation}\label{edp_superficies_minimas}
		(1 + f_y^2) f_{xx}  - 2 f_x f_y f_{xy} + (1+f_x^2) f_{yy} = 0
	\end{equation}
	
	que é uma EDP de 2da ordem.
	
	Um exemplo trivial da equação \ref{edp_superficies_minimas} é a função linear
	\begin{equation*}
		f(x,y) = ax + by + c,
	\end{equation*}
	
	como $a, b, c \in \realnumbers$.
\end{exemplo}

\begin{exemplo}[Superfície de Scherk]
	Suponha que
	\begin{equation*}
		f(x,y) = g(x) + h(y).
	\end{equation*}
	
	Neste caso, a equação \ref{edp_superficies_minimas} pode ser escrita como
	\begin{equation*}
		(1 + (h')^2(y)) g''(x) + (1 + (g')^2(x)) h''(y) = 0,
	\end{equation*}
	
	ou seja
	\begin{equation*}
		\frac{g''(x)}{1 + (g')^2(x)} + \frac{h''(y)}{1 + (h')^2(y)} = 0.
	\end{equation*}
	
	Isso implica
	\begin{equation*}
		\frac{g''(x)}{1 + (g')^2(x)} = - \frac{h''(y)}{1 + (h')^2(y)} = \text{constante}.
	\end{equation*}
	
	Integrando, obtemos (a menos de constantes) que
	\begin{align*}
		g(x) &= \ln (\cos x)\\
		h(y) &= -\ln (\cos y).
	\end{align*}
	
	A menos de dilatações e translações uma parte da superfície pode ser representada pelo gráfico da função
	\begin{equation*}
		\ln \left( \frac{\cos x}{\cos y} \right), 0 < x,y < \frac{\pi}{2}
	\end{equation*}
\end{exemplo}