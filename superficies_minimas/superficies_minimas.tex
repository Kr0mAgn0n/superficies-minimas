\section{Superfícies Mínimas}

\begin{defi}
	Uma superfície regular $M$ em $\realnumbers^3$ é chamada \emph{superfície mínimas} se $H(p)=0$ para qualquer $p \in M$.
\end{defi}

\begin{obse}
	Se $H \equiv 0$, então $K_1 + K_2 \equiv 0$. Logo $K_1 = -K_2$
\end{obse}

\begin{exemplo}
	Um plano em $\realnumbers^3$ é trivialmente mínima, pois $K_1=K_2=0$.
\end{exemplo}

A motivação histórica do estudo das superfícies mínimas foi dada por Lagrange o ano 1760 como seguinte problema:

Dado uma curva fechada $\gamma$ em $\realnumbers^3$, sem autointerseções, determinar a superfície de área mínima, e que tem $\gamma$ como fronteira.

Seja $M$ uma superfície regular orientada em $\realnumbers^3$, e considere uma função $f \in \smoothfunction{M}$.

\begin{defi}
	Uma \emph{variação normal} de $M$, relativa à função $f$, é uma família de superfícies $M_t$, com $t \in (-\epsilon,\epsilon)$, dadas por:
	\begin{equation*}
		p_t = p + t f(p) N(p),
	\end{equation*}
	
	onde $N$ é o campo unitário normal a $M$, na orientação de $M$.
	
\end{defi}

Para $\epsilon > 0$ suficientemente pequeno, cada conjunto $M_t$ também e uma superfície regular chamada uma \emph{superfície de variação}.

Note que para $t=0$, $M_0=M$. Se $f \equiv 1$, $M_t$ é uma superfície \emph{paralela} a $M$ a uma distancia $t$.

**gráfico**

Dados uma variação normal $M_t$ de $M$ relativa a uma função suave $f: M \rightarrow \realnumbers$, com $t \in (-\epsilon,\epsilon)$, e $D \subset M$ um domínio limitado, considere
\begin{equation*}
	D_t = \{ p_t \in M_t: p \in D \}
\end{equation*}

para cada $t \in (-\epsilon,\epsilon)$, $D_t$ é um domínio correspondente em $M_t$. Definimos em cada $t$
\begin{equation*}
	A(t) = \text{Area}(D_t)
\end{equation*}

\begin{teo}
	\begin{equation*}
		A'(0) = -2 \int_D Hf dA
	\end{equation*}
\end{teo}

A expressão acima chama-se a \emph{formula da primeira variação da área}.

\begin{proof}
	contenidos...
\end{proof}

\begin{prop}
	Uma superfície $M$ em $\realnumbers^3$ é mínima se e somente se $A'(0) = 0$.
\end{prop}

\begin{proof}
	contenidos...
\end{proof}

\begin{obse}
	Suponha que exista uma solução $M$ para o problema de Lagrange, e considere uma variação normal $M_t$ de $M$, com $t \in (-\epsilon,\epsilon)$, dada por uma função suave $f: M \rightarrow \realnumbers$ tal que $f_{|\partial M} = 0$. Como a área de $M$ é mínima temos, em particular, que
	\begin{equation*}
		A(t) \geq A(0)
	\end{equation*}
\end{obse}